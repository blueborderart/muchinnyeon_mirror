\usepackage{fontspec}
\defaultfontfeatures{Ligatures=TeX}
\setmainfont{Roboto}[Mapping=tex-text]
\usepackage{geometry}
\geometry{verbose,tmargin=1.5in,bmargin=1in,lmargin=1.5in,rmargin=1in}
\usepackage{fancyhdr}
\pagestyle{fancy}
\setcounter{secnumdepth}{4}
\setcounter{tocdepth}{4}
\usepackage{array}
\usepackage{longtable}
\usepackage{refstyle}
\usepackage{float}
\usepackage{todonotes}
\usepackage{graphicx}
\usepackage[toc, page]{appendix}
\ifdefined\HCode\else
\usepackage{xeCJK} % for cjk char
\setCJKmainfont{Noto Serif TC Regular}
% set different fonts for different sub CJK blocks
\xeCJKDeclareSubCJKBlock{Hangul}{"1100 -> "11FF, "3130 -> "318F, "A960 -> "A97F, "AC00 -> "D7AF, "D7B0 -> "D7FF}
\setCJKmainfont[Hangul]{Noto Serif KR Regular}

\makeatletter

%%%%%%%%%%%%%%%%%%%%%%%%%%%%%% LyX specific LaTeX commands.

\AtBeginDocument{\providecommand\figref[1]{\ref{fig:#1}}}
%% Because html converters don't know tabularnewline
\providecommand{\tabularnewline}{\\}
\RS@ifundefined{subsecref}
  {\newref{subsec}{name = \RSsectxt}}
  {}
\RS@ifundefined{thmref}
  {\def\RSthmtxt{theorem~}\newref{thm}{name = \RSthmtxt}}
  {}
\RS@ifundefined{lemref}
  {\def\RSlemtxt{lemma~}\newref{lem}{name = \RSlemtxt}}
  {}


%%%%%%%%%%%%%%%%%%%%%%%%%%%%%% Textclass specific LaTeX commands.
\newlength{\lyxlabelwidth}      % auxiliary length
\newcommand{\lyxaddress}[1]{
\par {\raggedright #1
\vspace{1.4em}
\noindent\par}
}

%%%%%%%%%%%%%%%%%%%%%%%%%%%%%% User specified LaTeX commands.
\setromanfont{Roboto}[Mapping=tex-text]
\setsansfont{Roboto}[Mapping=tex-text]
\setmonofont{Envy Code R}
\tolerance=270
\emergencystretch=1.5em
\date{}

\@ifundefined{showcaptionsetup}{}{%
 \PassOptionsToPackage{caption=false}{subfig}}
\usepackage{tocbibind}
\usepackage{subcaption}
\usepackage{booktabs}
\usepackage{colortbl}
\usepackage{xcolor}
\usepackage{multirow}
\usepackage{makecell}
\usepackage{pdflscape}
\usepackage[bottom]{footmisc} %fix footnote position
\usepackage{xfrac}
\usepackage{fontawesome}
\usepackage[normalem]{ulem}
\newcommand{\ra}[1]{\renewcommand{\arraystretch}{#1}}
\renewcommand{\tabcolsep}{7pt}
\usepackage[font={small,it}]{caption}
\makeatother
\usepackage{fancybox}
\usepackage[raggedright,bf,sf,toctitles]{titlesec}
\usepackage{listings}
\lstset{
basicstyle=\small\ttfamily,
columns=fullflexible,
breaklines=true,
breakindent=0pt,
frame=shadowbox
}
\usepackage{makeidx}
\makeindex
%use imakeidx to include index in toc
%\usepackage{imakeidx}
%\makeindex

\usepackage{epigraph}
\usepackage{xunicode}

\renewcommand{\numberline}[1]{#1~}

\usepackage{dialogue}
\usepackage{tikz}
\usetikzlibrary{shadows,calc}

\def\shadowshift{3pt,-3pt}
\def\shadowradius{6pt}

\colorlet{innercolor}{black!60}
\colorlet{outercolor}{gray!05}

\newcommand\drawshadow[1]{
    \begin{pgfonlayer}{shadow}
        \shade[outercolor,inner color=innercolor,outer color=outercolor] ($(#1.south west)+(\shadowshift)+(\shadowradius/2,\shadowradius/2)$) circle (\shadowradius);
        \shade[outercolor,inner color=innercolor,outer color=outercolor] ($(#1.north west)+(\shadowshift)+(\shadowradius/2,-\shadowradius/2)$) circle (\shadowradius);
        \shade[outercolor,inner color=innercolor,outer color=outercolor] ($(#1.south east)+(\shadowshift)+(-\shadowradius/2,\shadowradius/2)$) circle (\shadowradius);
        \shade[outercolor,inner color=innercolor,outer color=outercolor] ($(#1.north east)+(\shadowshift)+(-\shadowradius/2,-\shadowradius/2)$) circle (\shadowradius);
        \shade[top color=innercolor,bottom color=outercolor] ($(#1.south west)+(\shadowshift)+(\shadowradius/2,-\shadowradius/2)$) rectangle ($(#1.south east)+(\shadowshift)+(-\shadowradius/2,\shadowradius/2)$);
        \shade[left color=innercolor,right color=outercolor] ($(#1.south east)+(\shadowshift)+(-\shadowradius/2,\shadowradius/2)$) rectangle ($(#1.north east)+(\shadowshift)+(\shadowradius/2,-\shadowradius/2)$);
        \shade[bottom color=innercolor,top color=outercolor] ($(#1.north west)+(\shadowshift)+(\shadowradius/2,-\shadowradius/2)$) rectangle ($(#1.north east)+(\shadowshift)+(-\shadowradius/2,\shadowradius/2)$);
        \shade[outercolor,right color=innercolor,left color=outercolor] ($(#1.south west)+(\shadowshift)+(-\shadowradius/2,\shadowradius/2)$) rectangle ($(#1.north west)+(\shadowshift)+(\shadowradius/2,-\shadowradius/2)$);
        \filldraw ($(#1.south west)+(\shadowshift)+(\shadowradius/2,\shadowradius/2)$) rectangle ($(#1.north east)+(\shadowshift)-(\shadowradius/2,\shadowradius/2)$);
    \end{pgfonlayer}
}

\pgfdeclarelayer{shadow}
\pgfsetlayers{shadow,main}

\newsavebox\mybox
\newlength\mylen

\newcommand\shadowimage[2][]{%
\setbox0=\hbox{\includegraphics[#1]{#2}}
\setlength\mylen{\wd0}
\ifnum\mylen<\ht0
\setlength\mylen{\ht0}
\fi
\divide \mylen by 120
\def\shadowshift{\mylen,-\mylen}
\def\shadowradius{\the\dimexpr\mylen+\mylen+\mylen\relax}
\begin{tikzpicture}
\node[anchor=south west,inner sep=0] (image) at (0,0) {\includegraphics[#1]{#2}};
\drawshadow{image}
\end{tikzpicture}}

\usepackage{xifthen}

\makeatletter
\def\blfootnote{\xdef\@thefnmark{}\@footnotetext}
\makeatother

%best practice to is to load hyperref last
\usepackage[unicode=true,
bookmarks=true, bookmarksnumbered=true, bookmarksopen=false,
breaklinks=true, pdfpagelabels=true, pdfstartview={0 0 1}, pdfborder={0 0 0}, pdfborderstyle={}, linktocpage=true, hypertexnames=true]{hyperref}
\hypersetup{pdftitle={I'm the Murim's crazy bith}} 
% bookmark package to exclude a chapter from part
% have to be loaded after hyperref package
\usepackage{bookmark}
